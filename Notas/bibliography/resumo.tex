\documentclass[12pt,letterpaper]{article}

% just for the example
\usepackage{lipsum}
% Set margins to 1.5in
\usepackage[margin=1.5in]{geometry}

% for graphics
\usepackage{graphicx}

% for crimson text
\usepackage{crimson}
\usepackage[T1]{fontenc}

% setup parameter indentation
\setlength{\parindent}{0pt}
\setlength{\parskip}{6pt}

% for 1.15 spacing between text
\renewcommand{\baselinestretch}{1.15}

% For defining spacing between headers
\usepackage{titlesec}
% Level 1
\titleformat{\section}
  {\normalfont\fontsize{18}{0}\bfseries}{\thesection}{1em}{}
% Level 2
\titleformat{\subsection}
  {\normalfont\fontsize{14}{0}\bfseries}{\thesection}{1em}{}
% Level 3
\titleformat{\subsubsection}
  {\normalfont\fontsize{12}{0}\bfseries}{\thesection}{1em}{}
% Level 4
\titleformat{\paragraph}
  {\normalfont\fontsize{12}{0}\bfseries\itshape}{\theparagraph}{1em}{}
% Level 5
\titleformat{\subparagraph}
  {\normalfont\fontsize{12}{0}\itshape}{\theparagraph}{1em}{}
% Level 6
\makeatletter
\newcounter{subsubparagraph}[subparagraph]
\renewcommand\thesubsubparagraph{%
  \thesubparagraph.\@arabic\c@subsubparagraph}
\newcommand\subsubparagraph{%
  \@startsection{subsubparagraph}    % counter
    {6}                              % level
    {\parindent}                     % indent
    {12pt} % beforeskip
    {6pt}                           % afterskip
    {\normalfont\fontsize{12}{0}}}
\newcommand\l@subsubparagraph{\@dottedtocline{6}{10em}{5em}}
\newcommand{\subsubparagraphmark}[1]{}
\makeatother
\titlespacing*{\section}{0pt}{12pt}{6pt}
\titlespacing*{\subsection}{0pt}{12pt}{6pt}
\titlespacing*{\subsubsection}{0pt}{12pt}{6pt}
\titlespacing*{\paragraph}{0pt}{12pt}{6pt}
\titlespacing*{\subparagraph}{0pt}{12pt}{6pt}
\titlespacing*{\subsubparagraph}{0pt}{12pt}{6pt}

% Set caption to correct size and location
\usepackage[tableposition=top, figureposition=bottom, font=footnotesize, labelfont=bf]{caption}

% set page number location
\usepackage{fancyhdr}
\fancyhf{} % clear all header and footers
\renewcommand{\headrulewidth}{0pt} % remove the header rule
\rhead{\thepage}
\pagestyle{fancy}

% Overwrite Title
\makeatletter
\renewcommand{\maketitle}{\bgroup
   \begin{center}
   \textbf{{\fontsize{18pt}{20}\selectfont \@title}}\\
   \vspace{10pt}
   {\fontsize{12pt}{0}\selectfont \@author} 
   \end{center}
}
\makeatother

% Used for Tables and Figures
\usepackage{float}

% For using lists
\usepackage{enumitem}

% For full citations inline
\usepackage{bibentry}
\nobibliography*

% Custom Quote
\newenvironment{myquote}[1]%
  {\list{}{\leftmargin=#1\rightmargin=#1}\item[]}%
  {\endlist}
  
% Create Abstract 
\renewenvironment{abstract}
{\vspace*{-.5in}\fontsize{12pt}{12}\begin{myquote}{.5in}
\noindent \par{\bfseries \abstractname.}}
{\medskip\noindent
\end{myquote}
}

\begin{document}

% Set Title, Author, and emai
\title{Annotated Bibliography}
\author{Felipe Bailez \\ bailez@usp.br}

\maketitle
\thispagestyle{fancy}

\section*{Price discovery of the cryptocurrencies}

\subsection*{\bibentry{CHANG2020641}}
Esse artigo foi preparado com TeXstudio versão 2.11 e Overleaf. Deixe o primeiro parágrafo de uma seção, sub-seção ou item sem tabulação. Os demais parágrafos podem ser tabulados.  Deixe o texto dos parágrafos justificados, isto é, alinhados à esquerda e a direita, como já é feito automaticamente pelo LaTeX.    
Use apenas a fonte proporcional com serifas “Computer Modern Unicode Serif (CMU Serif)”, padrão do LaTeX, para escrever o texto do artigo.  O tamanho de fonte padrão é 12 pt.  Use esse tamanho em todo o corpo do artigo, a menos que outro tamanho seja indicado (como é o caso dos títulos de seções e sub-seções).

\subsection*{\bibentry{ALEXANDER2020100776}}
Esse artigo foi preparado com TeXstudio versão 2.11 e Overleaf. Deixe o primeiro parágrafo de uma seção, sub-seção ou item sem tabulação. Os demais parágrafos podem ser tabulados.  Deixe o texto dos parágrafos justificados, isto é, alinhados à esquerda e a direita, como já é feito automaticamente pelo LaTeX.    
Use apenas a fonte proporcional com serifas “Computer Modern Unicode Serif (CMU Serif)”, padrão do LaTeX, para escrever o texto do artigo.  O tamanho de fonte padrão é 12 pt.  Use esse tamanho em todo o corpo do artigo, a menos que outro tamanho seja indicado (como é o caso dos títulos de seções e sub-seções).
Esse artigo foi preparado com TeXstudio versão 2.11 e Overleaf. Deixe o primeiro parágrafo de uma seção, sub-seção ou item sem tabulação. Os demais parágrafos podem ser tabulados.  Deixe o texto dos parágrafos justificados, isto é, alinhados à esquerda e a direita, como já é feito automaticamente pelo LaTeX.    
Use apenas a fonte proporcional com serifas “Computer Modern Unicode Serif (CMU Serif)”, padrão do LaTeX, para escrever o texto do artigo.  O tamanho de fonte padrão é 12 pt.  Use esse tamanho em todo o corpo do artigo, a menos que outro tamanho seja indicado (como é o caso dos títulos de seções e sub-seções).


\subsection*{\bibentry{BRAUNEIS201858}}
Esse artigo foi preparado com TeXstudio versão 2.11 e Overleaf. Deixe o primeiro parágrafo de uma seção, sub-seção ou item sem tabulação. Os demais parágrafos podem ser tabulados.  Deixe o texto dos parágrafos justificados, isto é, alinhados à esquerda e a direita, como já é feito automaticamente pelo LaTeX.    
Use apenas a fonte proporcional com serifas “Computer Modern Unicode Serif (CMU Serif)”, padrão do LaTeX, para escrever o texto do artigo.  O tamanho de fonte padrão é 12 pt.  Use esse tamanho em todo o corpo do artigo, a menos que outro tamanho seja indicado (como é o caso dos títulos de seções e sub-seções).

\bibliographystyle{acm-sigchi} 
\nobibliography{bibtemp}


\end{document}
